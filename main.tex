\documentclass[a4,12pt, titlepage]{jarticle}
\title{少人数クラス内容報告(中間まとめ)・\\ 講義内容要約}
\author{ アドバイザー $\colon$ 岡田聡一教授 \\[50pt]
322301150 \quad 菊地雄大}
\date{}
\usepackage{amssymb}
\usepackage{amsmath}
\usepackage{amsthm}
\usepackage{mathrsfs}
\usepackage{ytableau}
\usepackage[left=30truemm,right=30truemm]{geometry}
\theoremstyle{definition}
\newtheorem{df}{定義}
\newtheorem{thm}{定理}
\newtheorem{prop}[thm]{命題}
\newtheorem{cor}[thm]{系}
\newtheorem*{prf}{証明}
\newtheorem{lem}[thm]{補題}
\newtheorem*{ex}{例}
\newtheorem*{re}{注意}


\begin{document}
\maketitle
%
%
\section{少人数クラス内容報告(中間まとめ)}
  M2の春学期の少人数クラスでは, 参考文献\cite{b1}の第1章から第3章まで読んだ.
%
\subsection{鏡映}
  この節では, 鏡映について定義する.
  $V$をユークリッド空間, すなわち, 内積$\langle$ , $\rangle$を伴った実ベクトル空間とする.
\bigskip

\begin{df}
  $ \alpha \in V $の鏡映$r_\alpha: V \to V $を
  \[
  r_\alpha (x) = x - \langle x , \alpha^{ \vee } \rangle, \quad \left( \alpha^{ \vee } = \frac{2 \alpha}{\langle \alpha, \alpha \rangle} \right)
  \]
  とする.
\end{df}

\begin{ex}
  $V = \mathbb{R}^2, \alpha = (1, 0)$ とする.
  
  このとき, 
  \[
  \alpha^{\vee} = \frac{2 \alpha}{\langle \alpha, \alpha \rangle} = \frac{2 (1, 0)}{1^2 + 0^2} = (2, 0)
  \]

  任意の $x = (x_1, x_2) \in \mathbb{R}^2$ に対する鏡映 $r_\alpha(x)$は次の通りである.
  \[
  \begin{aligned}
    r_\alpha(x) &= x - \langle x, \alpha^{\vee} \rangle \alpha \\
    &= (x_1, x_2) - \langle (x_1, x_2), (2, 0) \rangle (1, 0) \\
    &= (x_1, x_2) - 2x_1 (1, 0) \\
    &= (-x_1, x_2)
  \end{aligned}
  \]
  この鏡映は$x$ を $y$ 軸に関して反転させる操作になる.
\end{ex}

\begin{prop}
  鏡映 $r_\alpha: V \to V $ に対して, 次が成り立つ. 
  \begin{enumerate}
    \item $r_\alpha^2 = 1$ \quad (ただし, $1$ は恒等写像)
    \item $ \langle \alpha, \alpha^{\vee} \rangle = 2 $で, $r_\alpha(\alpha) = - \alpha$
    \item $V$ は $ \ker(r_\alpha - 1) \oplus \ker(r_\alpha + 1)$ と分解でき, $\ker(r_\alpha + 1)$ は$\alpha$を基底に持つ1次元空間である. 
    \item $\langle r_\alpha(x), r_\alpha(y) \rangle = \langle x, y \rangle $
  \end{enumerate}
\end{prop}

\subsection{ルート系とワイル群}
  この節では, ルート系とワイル群について定義する.
\bigskip

\begin{df} 
  $V$上のルート系$\Phi$を以下を満たす有限集合とする.
  \begin{enumerate}
    \item $ 0 \notin V , \emptyset \neq V $
    \item $ r_\alpha (\Phi) = \Phi \quad ( \alpha \in \Phi )$
    \item $ \langle \alpha, \beta^{ \vee } \rangle \in \mathbb{Z} \quad ( \alpha, \beta \in \Phi )$
    \item $ \beta \in \Phi $が$ \alpha \in \Phi $の倍数なら, $ \beta = \pm \alpha $
  \end{enumerate}
  このとき, $ \Phi $の元をルートという. \\
  また, $ \Phi^{ \vee } = \{ \alpha^{ \vee } | \alpha \in \Phi \} $の元をコルートという.
\end{df}

\begin{ex}
  \begin{enumerate}
    \item[]
    \item $V = \mathbb{R}^3$とする. $\Phi = \{ (\pm1, 0, 0) \}$はルート系である.
    \item $V = \mathbb{R}^2$とする. $\Phi = \{ (\pm \frac{1}{\sqrt{2}}, \pm \frac{1}{\sqrt{2}}) \}$はルート系である.
    このとき, $ \Phi^{\vee} = \left\{ (\pm \sqrt{2}, \pm \sqrt{2})\right\} $である.
  \end{enumerate}
\end{ex}

\begin{prop}
  \begin{enumerate}
    \item[]
    \item $\Phi = - \Phi$
    \item $(\alpha^{\vee})^{ \vee} = \alpha$
    \item $\Phi^{\vee}$もルート系になる.
  \end{enumerate}
\end{prop}

\begin{df}  
  \begin{enumerate} 
    \item[]
    \item $ \Phi $が可約であるとは, $ \Phi = \Phi_1 \cup \Phi_2 $で,任意の$ x \in \Phi_1, y \in \Phi_2 $で,
    $ \langle x, y \rangle = 0 $となるルート系$ \Phi_1, \Phi_2  $が存在するときをいう.
    \item $ \Phi $が既約(または, 単純)であるとは, $ \Phi $が可約ではないときをいう.
    \item $ \Phi $がsimply-lacedであるとは, 全てのルートの長さが同じであるときをいう.
  \end{enumerate} 
\end{df}

\begin{ex}
  \begin{enumerate}
    \item[]
    \item $ V = \mathbb{R}^2$とする.$ \Phi = \{ (\pm 1, 0), (0, \pm1) \}$は可約なルート系である.実際,
    $ \Phi_1 = \{ (\pm 1, 0)\}$, $ \Phi_2 = \{ (0, \pm 1)\}$と分けられる.
    \item $ V = \mathbb{R}^2$とする. $ \Phi = \{ (\pm \sqrt{2}, 0), (0, \pm\sqrt{2}), (\pm\sqrt{2}, \pm \sqrt{2}) \}$は
    $(\sqrt{2}, 0)$の長さが$\sqrt{2}$, $(\sqrt{2}, \sqrt{2})$の長さが2であるから, simply-lacedでないルート系になる.
  \end{enumerate}
\end{ex}

\begin{df}
  $\Phi$と交わらない原点を通る超平面を固定する. 
  \begin{enumerate}
    \item この超平面にある1つの側にあるルートを正ルート, もう1つの側にあるルートを負ルートとよぶ.
    \item 正ルート全体を$ \Phi^{+} $, 負ルート全体を$ \Phi^{-} $で表す. 
    \item $\alpha \in \Phi^{+} $が単純であるとは, $\alpha$が他の正ルートの和で表せないときをいう.
  \end{enumerate}
\end{df}

\begin{ex}
  $ V = \mathbb{R}^2$とする. $ \Phi = \{ (\pm \sqrt{2}, 0), (0, \pm\sqrt{2}), (\pm\sqrt{2}, \pm\sqrt{2}) \}$とする. \\
  例えば, 正ルートは $\{ ( \sqrt{2}, 0), (0, \sqrt{2}), (\sqrt{2}, \sqrt{2}), (-\sqrt{2}, \sqrt{2}) \}$, 負ルートは \\
  $\{ ( -\sqrt{2}, 0), (0, -\sqrt{2}), (\sqrt{2}, -\sqrt{2}), (-\sqrt{2}, -\sqrt{2}) \}$と分けられる.  \\
  このとき, 単純な正ルート全体は,  $\{ ( \sqrt{2}, 0), (0, \sqrt{2}) \}$や$ \{ (\sqrt{2}, \sqrt{2}), (-\sqrt{2}, \sqrt{2}) \}$などとして,
  取れる.
\end{ex}

\begin{re}
  正ルートや単純な正ルートの定め方は, 一意に定まらない. 一つ固定して考える.
\end{re}

\begin{prop}
  $\Sigma$を単純な正ルートのなす集合とする.
  \begin{enumerate} 
    \item $\Sigma$の元は,線型独立.
    \item $\alpha \in \Sigma, \beta \in \Phi^{+}$なら, $\alpha = \beta$または$r_{\alpha}(\beta) \in \Phi^{+}$
    \item $\alpha, \beta \in \Sigma $で, $\alpha \neq \beta $なら, $ \langle \alpha, \beta \rangle \leq 0 $
    \item 任意の$\alpha \in \Phi^{+}$は
    $$ \alpha = \sum_{ \beta \in \Sigma} n_{\beta} \beta \quad (n_{\beta} \geq 0, n_{\beta} \in \mathbb{Z}) $$
  \end{enumerate}
\end{prop}

\begin{df}
  $I = \{ 1, 2, \cdots, r \}$を添字集合とし, $\Sigma = \{ \alpha_i \mid i \in I \}$とする. \\
  $ i \in I $に対し, この鏡映を$s_i$で表す.これを単純鏡映という.
\end{df}

\begin{prop}
  $ i \in I, \alpha \in \Phi^{+}$とする. このとき,
  $\alpha = \alpha_i$か$s_i(\alpha) \in \Phi^{+}$のいずれかである. \\
  よって, $s_i$は$\Phi^{+} \backslash \{ \alpha_i\} $上を置換する.
\end{prop}

\begin{df}
  $W = \langle r_\alpha \mid \alpha \in \Phi \rangle $を$\Phi$のワイル群という.
\end{df}

\begin{ex}
  $V = \mathbb{R}^2, \Phi = \{ (\pm 1, 0), (0, \pm 1) \}$とする. \\
  $r_{ (1, 0)} =(x, y) = (-x, y), r_{ (0 ,1)} =(x, y) = (x, -y)$である.
  $r_{ (1, 0)} \circ r_{ (0 ,1)}(x, y) = r_{ (0, 1)} \circ r_{ (1 ,0)}(x, y) = (-x, -y)$となる.
  これらの元は全て位数2である.よって, 
  $$ W = \{ 1, r_{ (1, 0)}, r_{ (0 ,1)}, r_{ (1, 0)} \circ r_{ (0 ,1)} \}$$
  であり, $C_2 \times C_2 $(ただし, $C_2$は位数2の巡回群)に群同型である.
\end{ex}

\subsection{weight lattice}
  この節では,weight latticeについて定義する.
\bigskip

\begin{df}
  $\Phi$を$V$におけるルート系とする. \\
  weight latticeとは, $V$を生成するlattice(自由$\mathbb{Z}$加群)$\Lambda$で以下を満たすときをいう.
  \begin{enumerate}
    \item $\Phi \subset \Lambda $
    \item 任意の$ \lambda \in \Lambda, \alpha \in \Phi $で, $ \langle \lambda, \alpha^{ \vee } \rangle \in \mathbb{Z} $
  \end{enumerate}
  この元をweightという.
\end{df}

\begin{df}
  \begin{enumerate}
    \item[]
    \item weight latticeが半単純であるとは, $\Phi$が$V$を張るときをいう.
    \item root lattice $\Lambda_{ root }$を$\Phi$によって張られた空間とする.
  \end{enumerate}
\end{df}

\begin{ex}
  $V = \mathbb{R}^3$とする. $\Phi = \{ (\pm1, 0, 0) \}$とする. $\Lambda = \mathbb{Z}^3$はweight latticeになる.
  $\Phi$は$V$を張らないので, 半単純ではない. また, $\Lambda_{ root } = \{ (x, 0 , 0) \mid x \in \mathbb{Z} \}$である.
\end{ex}

\begin{df}
  \begin{enumerate}
    \item []
    \item $\Lambda$上に順序 $\lambda \leq \mu$ を
      $$ \lambda - \mu = \sum_{i \in I} c_i \alpha_i \quad ( c_i \geq 0 ) $$
      と定める.
    \item $\lambda^{+} = \{ \lambda \mid \langle \lambda, \alpha_i^{\vee} \rangle \geq 0 \quad (i \in I) \}$ とし, この元を dominant weight という.
    \item $\lambda$が $\langle \lambda, \alpha_i^{\vee} \rangle > 0 \quad (i \in I)$ ならば, strictly dominant weight という.
    \item $\bar{\omega_i}$ が fundamental weight であるとは, $\langle \bar{\omega_i}, \alpha_j^{\vee} \rangle = \delta_{i, j}$
    であるときをいう.
  \end{enumerate}
\end{df}

この節の最後に, 重要なcrystalを紹介しよう.

\begin{ex}
  $V = \mathbb{R}^{r+1}$ とする. ルート系を $\Phi = \{ e_i - e_j \mid i \neq j \}$ とし, $\Phi^{+} = \{ e_i - e_j \mid i < j \}$ とする.
  これは既約で, simply-laced である. このとき, 単純な正ルート全体は $\Sigma = \{ \alpha_i = e_i - e_{i+1} \mid 1 \leq i \leq r \}$ となる. \\
  weight latticeは $\Lambda = \mathbb{Z}^{r+1}$ とする. これは半単純ではない. \\
  $\lambda = (\lambda_1, \lambda_2, \cdots, \lambda_{r+1})$ が dominant であることと $\lambda_1 \geq \lambda_2 \geq \cdots \geq \lambda_{r+1}$
  が成り立つことは必要十分である. また, fundamental weight は $\omega_i = e_1 + e_2 + \cdots + e_i$ である.
\end{ex}


%
\subsection{Kashiwara crystals}
この節では, Kashiwara crystalsについて定義する. \\
$\mathbb{Z} \cup \{ - \infty \} $に $- \infty < n ( n \in \mathbb{Z} ) $と順序を入れ, $ - \infty + n = - \infty ( n \in \mathbb{Z} ) $
と定義する.

\begin{df}
  添字集合 $I$ をともなったルート系 $\Phi$ と weight lattice $\Lambda$ を固定する.
  タイプ $\Phi$ の Kashiwara crystal は次の写像をともなった空でない集合 $\mathscr{B}$ である.
  \begin{enumerate}
    \item $e_i, f_i : \mathscr{B} \to \mathscr{B} \cup \{ 0 \}$
    \item $\epsilon_i, \phi_i : \mathscr{B} \to \mathbb{Z} \sqcup \{ - \infty \}$
    \item $\mathrm{wt} : \mathscr{B} \to \Lambda$
  \end{enumerate}
  で, 次の(A1), (A2)を満たす.
  \begin{enumerate}
    \item [(A1)] 任意の$x, y \in \mathscr{B}$に対し, $e_i(x) = y $であることと, $f_i(y) = x $であることは必要十分条件である.
    このとき, 
    \begin{enumerate}
      \item [] $\mathrm{wt}(y) = \mathrm{wt}(x) + \alpha_i$
      \item [] $\epsilon(y) = \epsilon(x) - 1$
      \item [] $\phi_i(y) = \phi_i(x) + 1 $
    \end{enumerate}
    が成り立つ.
    \item [(A2)] $\phi_i(x) = \langle \mathrm{wt}(x), \alpha_i ^{ \vee } \rangle + \epsilon_i(x) $
    が成り立つ.\\
    特に, $\Phi_i(x) = - \infty$なら, $\epsilon_i(x) = - \infty $である.このとき, $e_i(x) = f_i(x) = 0$を仮定する.
  \end{enumerate}
\end{df}

\begin{df} 上記の定義において,
  \begin{enumerate}
    \item crystal$\mathscr{B}$の元の個数を次数という.
    \item 写像$\mathrm{wt}$をweight写像という.
    \item $e_i, f_i$をkashiwara(または, cryastal)作用素という.
    \item $\phi_i, \epsilon_i$はstring lengthと呼ばれることもある.
    \item $\phi_i, \epsilon_i$が$- \infty $の値を取らないとき, $\mathscr{B}$は有限なタイプであるという.
    \item $\phi_i(x) = \max\{ k \in \mathbb{Z}_{\geq 0} \mid f_i^k(x) \neq 0 \}, \quad \epsilon_i(x) = \max\{ k \in \mathbb{Z}_{\geq 0} \mid e_i^k(x) \neq 0 \}$ \\
    が成り立つとき, $\mathscr{B}$はseminomarlという.
    \item $\phi_i, \epsilon_i$が非負の値を持つとき, $\mathscr{B}$はupper seminomarlであるという.
    \item 任意の$i \in I$で$e_i(u) = 0$となる元$u \in \mathscr{B}$をhighest weight元という. このとき, $\mathrm{wt}(u)$をhighest weightという.
  \end{enumerate}
\end{df}

\begin{prop}
  ルート系が半単純で, $\mathscr{B}$が有限なタイプのcrystalと仮定する. このとき,
  $$ \mathrm{wt}(x) = \sum_{ i \in I} (\varphi_i(x) - \varepsilon_i(x))\bar{\omega_i}$$
  が成り立つ.
\end{prop}

\begin{prop}
  $\mathscr{B}$をseminomaralなcrystalとする. $u$をhighest weight元とする. このとき, $\mathrm{wt}(u)$はdominantである.
\end{prop}

\begin{prop}
  $\mathscr{B}$をseminomaralなcrystalとする. $\mu, \nu \in \Lambda$をワイル群のある元$w$で, $w(\mu) = \nu$となる元とする.
  このとき,  
  \[
  \{ u \mid \mathrm{wt}(u) = \mu \} = \{ u \mid \mathrm{wt}(u) = \nu \}
  \]
  が成り立つ.
\end{prop}

\begin{df}
  \begin{enumerate}
    \item []
    \item $\mathscr{B}$をcrystalとする.このとき, $\mathscr{B}$上に頂点と$i \in I$でラベル付けられた辺を持つ有向グラフを対応できる. 
    $f_i(x) = y$のとき, $ x \xrightarrow{i} y$と書く.これを$\mathscr{B}$のcrystal graphという.
    \item $\mathscr{B}$ 上に, $x$ と $y$ が $y = f_i(x)$ または $x = e_i(y)$ を満たすとき, $x \sim y$ という同値関係を定める.
  \end{enumerate}
\end{df}

\begin{ex}
  タイプ$A_r$には,次のcrystal graphを持つ標準的なcrystalがある.
  $$\begin{ytableau} 1 \end{ytableau} \xrightarrow{1} \begin{ytableau} 2 \end{ytableau} \xrightarrow{2} \cdots \xrightarrow{r}
  \begin{ytableau} r \end{ytableau}$$
  $GL(r+1)$weight latticeを使い, $\mathrm{wt} \left( \vcenter{\hbox{\begin{ytableau} i \end{ytableau}}} \right) = e_i$と定める.
  さらに, seminomarlであるように$\varphi_i, \varepsilon_i$を定める. これを$\mathscr{B}_{(1)}$や$\mathbb{B}$で表す.
\end{ex}

\begin{ex}
  $\Lambda = \mathbb{Z}^n, n = r+1$とする.
  $\mathscr{B}_{(k)}$を分割$(k)$のsemistaandard tableau全体とする. その元を$ R = \begin{ytableau} j_1 & j_2 & \cdots & j_k  \end{ytableau}$
  (ただし, $j_1 \leq j_2 \leq \cdots \leq j_k \in [n]$)で表す. \\
  $\mathrm{wt}(R) = (\mu_i, \mu_2, \cdots, \mu_n)$(ただし, $\mu_i$は$R$の$i$の数)とする. 
  さらに, $\varphi_i(R)$を成分$j_1, j_2, \cdots, j_k$上の$i$の数, $\varepsilon_i(R)$を成分$j_1, j_2, \cdots, j_k$上の$i+1$の数とする. \\
  また, $\varphi_i(R) > 0$なら, $f_i(R)$を右端の$i$を$i+1$に変えて得られるタブロー, そうでないなら, $f_i(R) = 0$とする.
  同様に, $\varepsilon_i(R) > 0$なら, $f_i(R)$を左端の$i+1$を$i$に変えて得られるタブロー, そうでないなら, $f_i(R) = 0$とする.\\
  これにより, $\mathscr{B}_{(k)}$はseminomarlなcrystalになる.
\end{ex}

%
\subsection{crystalのテンソル積と準同型}
この節では, Kashiwara crystalsのテンソル積と準同型について定義する. 
\bigskip

\begin{df}
  $\mathscr{B}, \mathscr{C}$を同じルート系 $\Phi$ のcrystalとする。
  $\mathscr{B} \otimes \mathscr{C}$ を次のように定める.
  \begin{enumerate}
    \item $\mathrm{wt}(x \otimes y) = \mathrm{wt}(x) + \mathrm{wt}(y)$
    \item $f_i(x \otimes y) = 
    \begin{cases} 
      f_i(x) \otimes y & \text{if } \varphi_i(y) \leq \varepsilon_i(x) \\
      x \otimes f_i(y) & \text{if } \varphi_i(y) > \varepsilon_i(x)
    \end{cases}$
    \item $e_i(x \otimes y) = 
    \begin{cases} 
      e_i(x) \otimes y & \text{if } \varphi_i(y) < \varepsilon_i(x) \\
      x \otimes e_i(y) & \text{if } \varphi_i(y) \geq \varepsilon_i(x)
    \end{cases}$
    \item $x \otimes 0 = 0 \otimes x = 0$
    \item $\varphi_i(x \otimes y) = \varphi_i(x) + \max\{ \varphi_i(x), \varphi(y) + \langle \mathrm{wt}(x), \alpha_i^{ \vee } \rangle \} $
    \item $\varepsilon_i(x \otimes y) = \varepsilon_i(y) + \max\{ \varepsilon_i(y), \varepsilon(x) - \langle \mathrm{wt}(y), \alpha_i^{ \vee } \rangle \} $
  \end{enumerate}
\end{df}

\begin{prop}
  $\mathscr{B} \otimes \mathscr{C}$はcrystalである. さらに, $\mathscr{B}, \mathscr{C}$がseminomarlなら, $\mathscr{B} \otimes \mathscr{C}$もseminomarlである. 
\end{prop}

\begin{df}
  $\mathscr{B}と\mathscr{C}$をルート系$\Phi$, 添字集合$I$を持つcrystalとする. \\
  写像$ \psi : \mathscr{B} \to \mathscr{C} \sqcup \{ 0 \}$がcrystal準同型であるとは, 次を満たすときをいう. \\
  1. $b \in B$ かつ $\psi(b) \in C$ であるとき, 
    \begin{enumerate}
      \item[a] $\mathrm{wt}(\psi(b)) = \mathrm{wt}(b)$
      \item[b] $\epsilon_i(\psi(b)) = \epsilon_i(b)$ for all $i \in I$
      \item[c]$\phi_i(\psi(b)) = \phi_i(b)$ for all $i \in I$
    \end{enumerate}
  2. $b, e_i b \in B$ かつ $\psi(b), \psi(e_i b) \in C$ であるとき, $\psi(e_i b) = e_i(\psi(b))$ である. \\
  3. $b, f_i b \in B$ かつ $\psi(b), \psi(f_i b) \in C$ であるとき, $4(f_i b) = f_i(\psi(b))$ である.
\end{df}

\begin{df}
準同型 $\psi$ が任意の $i \in I$ に対して $e_i$ および $f_i$ と可換であるとき, $\psi$ は strict であるという.
また, crystal準同型 $\psi : B \to C \sqcup \{ 0 \}$ がcrystal同型であるとは, 誘導される写像 $\psi : B \sqcup \{ 0 \} \to C \sqcup \{ 0 \}$ で $\psi(0) = 0$ を満たすものが全単射である場合をいう.
\end{df}

\begin{prop}
  $\mathscr{B}, \mathscr{C}, \mathscr{D}$ をcrystalとする.
  このとき, $(\mathscr{B} \otimes \mathscr{C}) \otimes \mathscr{D}$と$\mathscr{B} \otimes (\mathscr{C} \otimes \mathscr{D})$は同型になる. 

\end{prop}

%
\subsection{タブローのクリスタル}
この節ではタブローのクリスタルについて説明する. \\
$ [n] = \{ 1, 2, \cdots , n \} $とする.
$k$を正整数とし, $\lambda$を$k$の分割とする.
形$\lambda$のヤングタブロー全体を$YD(\lambda)$で表す.
形$\lambda$のsemistaandard tableu全体を$\mathscr{B_\lambda}$とする.
\bigskip

\begin{df}
  写像 $RR : \mathscr{B}_{(k)} \to \mathbb{B}^{ \otimes k}$ を次のように定める. 
  \[
  RR \left( \vcenter{\hbox{\begin{ytableau} i_1 & i_2 & \cdots & i_k \end{ytableau}}} \right) = 
    \vcenter{\hbox{\begin{ytableau} i_1 \end{ytableau}}} \otimes 
    \vcenter{\hbox{\begin{ytableau} i_2 \end{ytableau}}} \otimes 
    \cdots \otimes 
    \vcenter{\hbox{\begin{ytableau} i_k \end{ytableau}}}
  \]
\end{df}  

\begin{prop}
  写像 $RR$は$\mathscr{B}_{(k)}$から$\mathbb{B}^{ \otimes k}$への準同型である.
\end{prop}

\begin{df}
  定義した写像 $R \to RR(R)$を、すべての形 $\lambda$ のsemistaandard tableu$T$ への写像に以下のように拡張する. \\
  この写像も $T \to RR(T)$ と表し, $RR(T)$ を$T$ の各行を順に読み出し, その順序は下から上に向かって行を取るようにする.
  これを\textit{row reading}という.
\end{df} 
 
\begin{ex}
  \[
  T = \vcenter{\hbox{\begin{ytableau} 
    2 & 2 & 2 & 3 & 4 \\ 
    2 & 2 & 3 \\ 
    3 & 5 
  \end{ytableau}}}
  \]
  とする. このとき,
  \begin{align*}
  RR(T) &= RR\left( \vcenter{\hbox{\begin{ytableau}  3 & 5 \end{ytableau}}} \right) \otimes RR\left( \vcenter{\hbox{\begin{ytableau}  2 & 2 & 3 \end{ytableau}}} \right)
  \otimes RR\left( \vcenter{\hbox{\begin{ytableau}  2 & 2 & 2 & 3 & 4 \end{ytableau}}} \right) \\
  &= RR\left( \vcenter{\hbox{\begin{ytableau}  3 \end{ytableau}}} \right) \otimes RR\left( \vcenter{\hbox{\begin{ytableau}  5 \end{ytableau}}} \right) \otimes
  RR\left( \vcenter{\hbox{\begin{ytableau}  2 \end{ytableau}}} \right) \otimes RR\left( \vcenter{\hbox{\begin{ytableau}  2 \end{ytableau}}} \right) \otimes
  RR\left( \vcenter{\hbox{\begin{ytableau}  3 \end{ytableau}}} \right) \otimes RR\left( \vcenter{\hbox{\begin{ytableau}  2 \end{ytableau}}} \right) \\
  &\quad \otimes
  RR\left( \vcenter{\hbox{\begin{ytableau}  2 \end{ytableau}}} \right) \otimes RR\left( \vcenter{\hbox{\begin{ytableau}  2 \end{ytableau}}} \right) \otimes 
  RR\left( \vcenter{\hbox{\begin{ytableau}  2 \end{ytableau}}} \right) \otimes RR\left( \vcenter{\hbox{\begin{ytableau}  3 \end{ytableau}}} \right) \otimes
  RR\left( \vcenter{\hbox{\begin{ytableau}  4 \end{ytableau}}} \right)
  \end{align*}
  となる.
\end{ex}

\begin{ex}
  $n = 4, k= 5$とする.
  $5$の分割は, $(5), (4, 1), (3, 2), (3, 1, 1), (2, 2, 1), (2, 1, 1, 1), (1, 1, 1, 1, 1)$である. また, $GL(4)$crystalは,
  $$\begin{ytableau} 1 \end{ytableau} \xrightarrow{1} \begin{ytableau} 2 \end{ytableau} \xrightarrow{2} \begin{ytableau} 3 \end{ytableau} \xrightarrow{3}
  \begin{ytableau} 4 \end{ytableau} \xrightarrow{4} \begin{ytableau} 5 \end{ytableau}$$
\end{ex}
  
\begin{ex}
  $n = 2, k= 3$とする.
  $3$の分割は, $(3), (2, 1), (1, 1, 1)$である. また, $GL(2)$crystalは,
  $$\begin{ytableau} 1 \end{ytableau} \xrightarrow{1} \begin{ytableau} 2 \end{ytableau}$$
  $$\begin{ytableau} 1 \end{ytableau}  \otimes \begin{ytableau} 1 \end{ytableau} \otimes \begin{ytableau} 1 \end{ytableau}
  \xrightarrow{1} \begin{ytableau} 1 \end{ytableau}  \otimes \begin{ytableau} 1 \end{ytableau} \otimes \begin{ytableau} 2 \end{ytableau}
  \xrightarrow{1} \begin{ytableau} 1 \end{ytableau}  \otimes \begin{ytableau} 2 \end{ytableau} \otimes \begin{ytableau} 2 \end{ytableau}
  \xrightarrow{1} \begin{ytableau} 2 \end{ytableau}  \otimes \begin{ytableau} 2 \end{ytableau} \otimes \begin{ytableau} 2 \end{ytableau}
  $$
  $$\begin{ytableau} 2 \end{ytableau}  \otimes \begin{ytableau} 1 \end{ytableau} \otimes \begin{ytableau} 1 \end{ytableau}
  \xrightarrow{1} \begin{ytableau} 2 \end{ytableau}  \otimes \begin{ytableau} 1 \end{ytableau} \otimes \begin{ytableau} 2 \end{ytableau}
  $$ 
  $$\begin{ytableau} 1 \end{ytableau}  \otimes \begin{ytableau} 2 \end{ytableau} \otimes \begin{ytableau} 1 \end{ytableau}
  \xrightarrow{1} \begin{ytableau} 2 \end{ytableau}  \otimes \begin{ytableau} 2 \end{ytableau} \otimes \begin{ytableau} 1 \end{ytableau}
  $$ 
\end{ex}

\begin{ex}


\end{ex} 



\begin{thebibliography}{99}
\bibitem{b1} Daniel Bump, Anne Schilling「CRYSTAL BASES Representations and Combinatorics」World Scientific, 2017.
\end{thebibliography}

%
\section{講義内容要約}
%
なし
%
\end{document}
